\documentclass[11pt,fullpage]{article}
\usepackage[urlcolor=blue,colorlinks=true]{hyperref}

\begin{document}

\title{Using IntelliJ with the Broad Sting Repository}
\author{Aaron Mckenna}
\date{\today}
\maketitle

\section{Overview}
Using JetBrains IntelliJ with the Broad String repository is a relatively simple process.  

\subsection{Getting the source}
The source can be checked out from the repository, from the following link: \\ \\
\url{https://svnrepos/sting}\\ \\
Using the command:\\ \\
\texttt{svn co http://svnrepos/sting ./sting} \\ \\ 
Replacing the second directory with the location you'd like the code to reside in on your local machine.  
\subsection{Getting IntelliJ}
Licensing of JetBrains IntelliJ at MIT is done one license at a time, we don't have a site license for it.  You'll need to contact help,
(help@broad.mit.edu) and they'll retrieve a license for you.  In the the mean time you can download the IntelliJ demo, which is a 
fully featured 30 day trial that the license can be entered into.  You can download it from their site: \\ \\
\url{http://www.jetbrains.com/idea/download} \\ \\
When you do have your license, you can enter the license into IntelliJ by selecting help, and then the register drop down option. 
\section{Working with the code in IntelliJ}

\subsection{Setting up the project}
IntelliJ doesn't need to know that the project is an ant project specifically, instead it knows how to build an ant project once it's loaded.  To
setup a new project, create a new project, selecting the \textit{Create project from existing sources} option.  It's best to work with either the 
playground or the core sources one at a time, and all development work should be done in the playground.  The best option in IntelliJ is to 
choose the java directory in playground as your source, and let IntelliJ create the project directory there. 
\\
\begin{enumerate}
\item Choose \textit{New Project} from the file menu \\
\item Choose the \textit{Create project from existing sources} option.\\
\item Name your project, something sensical like Playground, and choose the source directory \textit{sting/playground/java}. \\
\item Choose the .idea (directory format) project storage.  This is just a recommendation, you can do either. \\
\item After clicking next, IntelliJ should detect that the java directory already has a src directory, click through the next few windows and finally select finish.
\end{enumerate}

To have IntelliJ recognize that the project is an Ant project, select the Ant tab on the far right of the IntelliJ window.  When the window pane opens, click the
plus symbol, and select the build.xml in the /playground/java directory.  Now the Ant build target window should have a list of the targets that were loaded
from the build file.

\subsection{Building the code}
To build the code, select the compile list options from the Ant window.  Ivy should automatically build the dependancy list, and fetch any libraries that aren't already loaded.  To 
transfer the code to one of our machines once a jar file is created.  

Useful targets include:
\begin{description}
  \item[compile] Compiles all java code in the source tree.  Places generated classes in the build directory.
  \item[dist] Generates jar files, suitable for running via java -jar {YOUR\_JAR}.  Places resulting jars in the dist subdirectory.
  \item[resolve] Resolves third-party dependencies.  Downloads all third-party dependencies to the lib directory.
  \item[javadoc] Generates javadoc for the source tree.  Places javadoc in the javadoc directory.
  \item[clean] Removes artifacts from old compilations / distributions.
\end{description}
All available ant targets can be viewed by running 'ant -projecthelp' in the directory containing build.xml.

\subsection{Setting up the header to contain the license file}

\section{Platform-specific Notes}

\subsection{Using IntelliJ on Linux}
If IntelliJ hangs or crashes, try changing the default arguments specified in the \$\{INTELLIJ\_HOME\}/idea.vmoptions file to the following:

\begin{verbatim}
-server
-Xms768m
-Xmx1248m
-Xmn170m
-XX:MaxPermSize=300m
-ea
\end{verbatim}


\end{document}
