\documentclass[11pt,fullpage]{article}
\usepackage[urlcolor=blue,colorlinks=true]{hyperref}

\begin{document}

\title{The Sting Build Environment}
\author{Matt Hanna}
\date{16 Mar 2009}
\maketitle

\section{Getting the source}
gsa1 and gsa2 are earmarked for Sting repository development, and the scr1 thumper is used for Sting storage.

To download the source:
\begin{enumerate}
  \item log into gsa1 or gsa2.  Create a directory for yourself on the scr1 thumper as follows:
    \begin{verbatim}
      mkdir /wga/scr1/{YOUR USER NAME}
    \end{verbatim}
  \item Download the source as follows:
    \begin{verbatim}
      svn co https://svnrepos/Sting/trunk Sting
    \end{verbatim}
  \item (Optional) mount the thumper locally using share name smb://thumper01/scr1.
\end{enumerate}

\section{Build Prerequisites}
Sting requires Java SE 6 to compile.  The steps below describe preparing your system for JDK 1.6 compilation.

\subsection{Mac}
\begin{enumerate}
  \item Download the latest Mac Java service pack.  At the time of writing, the latest service pack is available here: \url{http://support.apple.com/downloads/Java_for_Mac_OS_X_10_5_Update_2}
  \item Set the JAVA\_HOME environment variable to the location of JDK1.6 (/System/Library/Frameworks/JavaVM.framework/Versions/1.6/Home).
\end{enumerate}

\subsection{Linux in the Broad Environment}
  To compile Sting on gsa1 or gsa2, add the following lines to your ~/.my.cshrc:
  \begin{verbatim}
    use -q Java-1.6
    use -q Ant-1.7
  \end{verbatim}

\section{Building the Source}
To build the source, locate all build.xmls for required projects.  In each directory containing a build.xml, run the command:
\begin{verbatim}
ant
\end{verbatim}

You might also find the following ant targets useful.  \begin{description}
  \item[compile] Compiles all java code in the source tree.  Places generated classes in the build directory.
  \item[dist] Generates jar files, suitable for running via java -jar {YOUR\_JAR}.  Places resulting jars in the dist subdirectory.
  \item[resolve] Resolves third-party dependencies.  Downloads all third-party dependencies to the lib directory.
  \item[javadoc] Generates javadoc for the source tree.  Places javadoc in the javadoc directory.
  \item[clean] Removes artifacts from old compilations / distributions.
\end{description}
View all available ant targets by running 'ant -projecthelp' in the directory containing build.xml.



\section{Adding Third-party dependencies}
A large number of popular third-party tools are available via the maven repository (\url{mvnrepository.com}).  If your tool is available in the maven repository, add a line to the ivy.xml file similar to the following:
\begin{verbatim}
    <dependency org="junit" name="junit" rev="4.4" />
\end{verbatim}
If your third-party dependency is not available via ivy, talk to Aaron or Matt.
\end{document}
