\documentclass[11pt,fullpage]{article}
\usepackage[urlcolor=blue,colorlinks=true]{hyperref}

\oddsidemargin 0.0in
\textwidth 6.5in

\begin{document}

\title{Genome Analysis Toolkit (GATK) User Manual}
\author{Matt Hanna}
\date{30 Mar 2009}
\maketitle

\section{Walkers}
\subsection{Command-line Arguments}
Users can create command-line arguments for walkers by creating public
member variables annotated with @Argument in the walker.  The
@Argument annotation takes a number of different parameters:

\begin{enumerate}
  \item [fullName] The full name of this argument.  Defaults to the
    toLowerCase()'d member name.  When specifying fullName on the 
    command line, prefix a double dash (--).
  \item [shortName] The alternate, short name for this argument.
    Defaults to the first letter of the member name.  When specifying 
    shortName on the command line, prefix a single dash (-).
  \item [doc] Documentation for this argument.  Will appear in help
    output when a user either requests help with the --help (-h)
    argument or when a user specifies an invalid set of arguments.
  \item [required] Whether the argument is required when used with
    this walker.  Default is required = true.
  \item [exclusive] Specifies that this argument is mutually
    exclusive of another argument in the same walker.  Defaults to not
    mutually exclusive of any other arguments.
  \item [defaultValue] Specifies the default value for this parameter,
    in string form.  
\end{enumerate}

\subsubsection{Passing Command-line Arguments}
Arguments can be passed to the walker using either the full name or
the short name.  If passing arguments using the full name, the syntax
is $--$$<$arg full name$>$ $<$value$>$.
\begin{verbatim}
--myint 6
\end{verbatim}

If passing arguments using the short name,
the syntax is -$<$arg short name$>$$<$value$>$.  Note that there is no space
between the short name and the value:
\begin{verbatim}
-m6
\end{verbatim}

Boolean (class) and boolean (primitive) arguments are a special in
that they require no argument.  The presence of a boolean indicates
true, and its absence indicates false.  The following example sets a
flag to true.
\begin{verbatim}
-B

\end{verbatim}

\subsubsection{Examples}

Create an required int parameter with full name --myint, short name
-m.
Pass this argument by adding ``--myint 6'' or -m6 to the command line.
\begin{verbatim}
import org.broadinstitute.sting.utils.cmdLine.Argument;

public class HelloWalker extends ReadWalker<Integer,Long> {
    @Argument
    public int myInt;
\end{verbatim}

Create an optional float parameter with full name
--myFloatingPointArgument, short name -m.  Pass this argument by
adding --myFloatingPointArgument 2.71 or -m2.71.
\begin{verbatim}
import org.broadinstitute.sting.utils.cmdLine.Argument;

public class HelloWalker extends ReadWalker<Integer,Long> {
    @Argument(fullName="myFloatingPointArgument",required=false,defaultValue="3.14159")
    public float myFloat;
\end{verbatim}

The GATK will parse the argument differently depending on the type of
the public member variable's type.  Many different argument types are 
supported, including primitives and their wrappers, arrays, typed and 
untyped collections, and any type with a String constructor.

When the GATK cannot completely infer the type (such as in the case of
untyped collections), it will assume that the argument is a String.
GATK is aware of concrete implementations of some interfaces and
abstract classes.  If the argument's member variable is of type List
or Set, the GATK will fill the member variable with a concrete
ArrayList or TreeSet, respectively.  Maps are not currently supported.

\subsection{Output}
By default, the walkers provide protected out and err PrintStreams.
Users can write to these streams just as they write to System.out and
System.err.  This output can be redirected to a file using the out,
err, and outerr command-line arguments.

\end{document}
